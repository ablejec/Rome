Preparation of reproducible statistical reports

Andrej Blejec

Preparation of various reports for communication of statistical information is a very common task. Reports are the final product of the sequence of steps consisting problem definition, data gathering and analysis, table and fugures organization and, finally typesetting of the report itself. In the usual framework, the steps are not connected: data source is cited, but the actual technical connection is not retained; it is difficult to figure out what type of analysis and formalism was used for analysis; tables and figures are copied and pasted without a trace. If some data change, become more complete or revised, it is necessary to prepare a new version of the report and to go again through all report making steps again, which can be tedious.
On the other hand, tools and systems to support preparation of reproducible statistical report are available and extensively used in some fields of scientific research. Reproducible reports consist of documents, which are a combination of text, meant for human readers, and computer code that is executed by computers. 